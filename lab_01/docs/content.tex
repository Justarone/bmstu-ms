\chapter{Задание}

\textbf{Цель работы:} построение гистограммы и эмпирической функции распределения.

\begin{enumerate}
    \item Для выборки объёма $n$ из генеральной совокупности $X$ реализовать в виде программы на ЭВМ
        \begin{enumerate}
            \item вычисление максимального значения $M_{\max}$ и минимального значения $M_{\min}$;
            \item размаха $R$ выборки;
            \item вычисление оценок $\hat\mu$ и $S^2$ математического ожидания $MX$ и дисперсии $DX$;
            \item группировку значений выборки в $m = [\log_2 n] + 2$ интервала;
            \item построение на одной координатной плоскости гистограммы и графика функции плотности распределения вероятностей нормальной случайной величины с математическим ожиданием $\hat{\mu}$ и дисперсией $S^2$;
            \item построение на другой координатной плоскости графика эмпирической функции распределения и функции распределения нормальной случайной величины с математическим ожиданием $\hat{\mu}$ и дисперсией $S^2$.
        \end{enumerate}
    \item Провести вычисления и построить графики для выборки из индивидуального варианта.
\end{enumerate}

\chapter{Теоретические сведения}

\section{Формулы для вычисления величин}

\subsection{Минимальное и максимальное значения выборки}
\begin{equation}
    \begin{aligned}
        M_{\max} = X_{(n)}\\
        M_{\min} = X_{(1)}
    \end{aligned}
\end{equation}

\subsection{Размах выборки}
\begin{equation}
    R = M_{\max} - M_{\min}.
\end{equation}

\subsection{Оценки математического ожидания и дисперсии}
\begin{equation}
    \begin{aligned}
    \hat\mu(\vec X_n) &= \frac 1n \sum_{i=1}^n X_i\\
    S^2(\vec X_n) &= \frac 1{n-1} \sum_{i=1}^n (X_i-\overline X_n)^2
    \end{aligned}
\end{equation}

\section{Определение эмпирической плотности и гистограммы}

Пусть $\vec x$ -- выборка из генеральной совокупности $X$. Если объем $n$ этой выборки велик, то значения $x_i$ группируют в интервальный статистический ряд. Для этого отрезок $J = [x_{(1)}, x_{(n)}]$ делят на $m$ равновеликих частей:

\begin{equation*}
    J_i = [x_{(1)} + (i - 1) \cdot \Delta, x_{(1)} + i \cdot \Delta), i = \overline{1; m - 1}
\end{equation*}

\begin{equation*}
    J_{m} = [x_{(1)} + (m - 1) \cdot \Delta, x_{(n)}]
\end{equation*}

\begin{equation*}
    \Delta = \frac{|J|}{m} = \frac{x_{(n)} - x_{(1)}}{m}
\end{equation*}

Интервальным статистическим рядом называют таблицу:

\begin{table}[htb]
    \centering
    \begin{tabular}{|c|c|c|c|c|}
        \hline
        $J_1$ & ... & $J_i$ & ... & $J_m$ \\
        \hline
        $n_1$ & ... & $n_i$ & ... & $n_m$ \\
        \hline
    \end{tabular}
\end{table}

где $n_i$ -- количество элементов выборки $\vec x$, которые $\in J_i$.

Обычно выборку разбивают на $m=[\log_2n]+2$ интервалов, где $n$ -- размер выборки.

Гистограмма -- это график эмпирической плотности. 

\textit{Эмпирической плотностью}, отвечающей выборке $\vec x$, называют функцию:
\begin{equation}
    \hat f(x) =
    \begin{cases}
        \frac{n_i}{n \Delta}, x \in J_i, i = \overline{1; m} \\
        0, \text{иначе} \\
    \end{cases}
\end{equation}

где $J_i$ -- полуинтервал статистического ряда, $n_i$ -- количество элементов выборки, входящих в полуинтервал, $n$ -- количество элементов выборки.


\section{Определение эмпирической функции распределения}

Пусть $\vec x = (x_1, ..., x_n)$ -- выборка из генеральной совокупности $X$. Обозначим $n(x, \vec x)$ -- число элементов вектора $\vec x$, которые имеют значения меньше $x$.

\textit{Эмпирической функцией распределения} называют функцию $F_n: \mathbb{R} \to \mathbb{R}$, определенную как: 

\begin{equation}
    F_n(x) = \frac{n(x, \vec x)}{n}
\end{equation}

\chapter{Результат работы}

\section{Код программы}

\lstinputlisting[style=mstyle]{../main.m}

\section{Результаты расчётов}

\begin{equation*}
    M_{\min} = -4,33 \\
\end{equation*}
\begin{equation*}
    M_{\max} = 0,01 \\
\end{equation*}
\begin{equation*}
    R = 4,34 \\
\end{equation*}
\begin{equation*}
    \hat\mu(\vec x_n) = 2,0585 \\
\end{equation*}
\begin{equation*}
    S^2(\vec x_n) = 0,994 \\
\end{equation*}
\begin{equation*}
    m = 8
\end{equation*}

\imgw{130mm}{f1}{Гистограмма и график функции плотности распределения вероятностей нормальной случайной величины с выборочными мат. ожиданием и дисперсией}

\imgw{130mm}{f2}{График эмперической функции распределения и функции распределения нормальной случайной величины с выборочными мат. ожиданием и дисперсией}
